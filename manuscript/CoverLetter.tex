% Cover letter using letter.sty
\documentclass[11pt, a4paper]{letter} % Uses 10pt
%Use \documentstyle[newcent]{letter} for New Century Schoolbook postscript font
% the following commands control the margins:
\topmargin=-1in    % Make letterhead start about 1 inch from top of page
\textheight=10in  % text height can be bigger for a longer letter
\oddsidemargin=0pt % leftmargin is 1 inch
\textwidth=6.8in   % textwidth of 6.5in leaves 1 inch for right margin
\setlength{\footskip}{1in}

\usepackage{sansmathfonts}
\usepackage{helvet}
\renewcommand\familydefault{\sfdefault}
\renewcommand{\rmdefault}{\sfdefault}
\usepackage{fontspec}
\usepackage[final]{pdfpages}


\usepackage{marvosym}
\usepackage{fontawesome}
\usepackage{xcolor}
\usepackage{hyperref}
\definecolor{blues}{HTML}{002F5D}
\definecolor{blues2}{HTML}{0068B4}
\definecolor{orange}{HTML}{EF7C00}
%\newcommand*\faMobile{{\FA\symbol{"F10B}}}
\newcommand*\varhrulefill[1][0.4pt]{\leavevmode\leaders\hrule height#1\hfill\kern0pt}


\usepackage{fontspec}
\setmainfont[Ligatures=TeX,BoldFont={Myriad Pro Semibold}]{Myriad Pro}

\renewcommand{\sfdefault}{\rmdefault}

\begin{document}

\signature{Dr Jack D. Evans}           % name for signature
\longindentation=0pt                       % needed to get closing flush left
\let\raggedleft\raggedright                % needed to get date flush left


\begin{letter}{ }
\noindent
\begin{minipage}[t]{.6\textwidth}
~\par\vspace{-\baselineskip}
\includegraphics[height=2cm]{TU_Dresden_Logo_HKS41.eps}
\vfill
\end{minipage}%
\begin{minipage}[t]{.4\textwidth}
{\large\bf Dr Jack D. Evans}\smallskip\\
Department of Inorganic Chemistry \\
Technische Universität Dresden \\
\faMobile~+49~163~3776138 \\
\Letter~jack.evans@tu-dresden.de
\end{minipage}%
\bigskip{\color{blues}\hrule}
\vfill % forces letterhead to top of page


\opening{To whom it may concern,}

Please find enclosed our manuscript entitled:

\smallskip
\begin{center}
\noindent \textbf{A universal standard archive file for adsorption data}

\noindent J. D. Evans, V. Bon, I. Senkovska, S. Kaskel
\end{center}
\smallskip

\noindent which we would like to submit for consideration as an \textit{Article} in \textit{Langmuir}.

In this article, we propose a new standard adsorption information file (AIF) for reporting adsorption isotherms, which builds upon other formats used throughout the literature, such as the crystallographic information file (CIF).
This universal archive file, AIF, is an easily extended free-format archive file, and this represents the first steps toward an open adsorption data format that can be used as a basis for a decentralized adsorption data library.
The use of open formats is a key requirement to facilitate the electronic transmission of adsorption data between laboratories, journals and larger databases in an effort to enhance the reproducibility of science in the field of porous materials.

To accompany this work, we have developed scripts and a computer program to generate an AIF from data in various formats produced by several instruments used by many laboratories around the world. This program and details for using the AIF format are freely available on a GitHub repository (\url{https://github.com/jackevansadl/adsorptioninformationformat}).

We believe this work is a springboard for an important discussion of the future of data reporting for porous materials. Thus, we are confident that this manuscript is appropriate and of interest to the readership of \textit{Langmuir}.

\sloppy{Please note, this article has been deposited at the \textit{ChemRxiv} preprint server with the DOI code: \mbox{\url{10.26434/chemrxiv.13562798}}.}

Thank you for your time and consideration of this manuscript.
\closing{Yours sincerely,}


%\encl{}  				% Enclosures


\vfill

\end{letter}

\end{document}
