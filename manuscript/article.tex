%%%%%%%%%%%%%%%%%%%%%%%%%%%%%%%%%%%%%%%%%%%%%%%%%%%%%%%%%%%%%%%%%%%%%
%% This is a (brief) model paper using the achemso class
%% The document class accepts keyval options, which should include
%% the target journal and optionally the manuscript type. 
%%%%%%%%%%%%%%%%%%%%%%%%%%%%%%%%%%%%%%%%%%%%%%%%%%%%%%%%%%%%%%%%%%%%%
\documentclass[journal=jacsat,manuscript=communication]{achemso}

%%%%%%%%%%%%%%%%%%%%%%%%%%%%%%%%%%%%%%%%%%%%%%%%%%%%%%%%%%%%%%%%%%%%%
%% Place any additional packages needed here.  Only include packages
%% which are essential, to avoid problems later. Do NOT use any
%% packages which require e-TeX (for example etoolbox): the e-TeX
%% extensions are not currently available on the ACS conversion
%% servers.
%%%%%%%%%%%%%%%%%%%%%%%%%%%%%%%%%%%%%%%%%%%%%%%%%%%%%%%%%%%%%%%%%%%%%
\usepackage[version=3]{mhchem} % Formula subscripts using \ce{}

%%%%%%%%%%%%%%%%%%%%%%%%%%%%%%%%%%%%%%%%%%%%%%%%%%%%%%%%%%%%%%%%%%%%%
%% If issues arise when submitting your manuscript, you may want to
%% un-comment the next line.  This provides information on the
%% version of every file you have used.
%%%%%%%%%%%%%%%%%%%%%%%%%%%%%%%%%%%%%%%%%%%%%%%%%%%%%%%%%%%%%%%%%%%%%
%%\listfiles

%%%%%%%%%%%%%%%%%%%%%%%%%%%%%%%%%%%%%%%%%%%%%%%%%%%%%%%%%%%%%%%%%%%%%
%% Place any additional macros here.  Please use \newcommand* where
%% possible, and avoid layout-changing macros (which are not used
%% when typesetting).
%%%%%%%%%%%%%%%%%%%%%%%%%%%%%%%%%%%%%%%%%%%%%%%%%%%%%%%%%%%%%%%%%%%%%
\newcommand*\mycommand[1]{\texttt{\emph{#1}}}

%%%%%%%%%%%%%%%%%%%%%%%%%%%%%%%%%%%%%%%%%%%%%%%%%%%%%%%%%%%%%%%%%%%%%
%% Meta-data block
%% ---------------
%% Each author should be given as a separate \author command.
%%
%% Corresponding authors should have an e-mail given after the author
%% name as an \email command. Phone and fax numbers can be given
%% using \phone and \fax, respectively; this information is optional.
%%
%% The affiliation of authors is given after the authors; each
%% \affiliation command applies to all preceding authors not already
%% assigned an affiliation.
%%
%% The affiliation takes an option argument for the short name.  This
%% will typically be something like "University of Somewhere".
%%
%% The \altaffiliation macro should be used for new address, etc.
%% On the other hand, \alsoaffiliation is used on a per author basis
%% when authors are associated with multiple institutions.
%%%%%%%%%%%%%%%%%%%%%%%%%%%%%%%%%%%%%%%%%%%%%%%%%%%%%%%%%%%%%%%%%%%%%
\author{Ruben Goeminne}
\affiliation[TU Ghent]
{Center for Molecular Modeling
Ghent University
Tech Lane Ghent Science Park Campus A, 9052 Zwijnaarde, Belgium}

\author{Simon Krause}
\affiliation[University of Groningen]
{Stratingh Institute for Chemistry
Faculty of Mathematics and Natural Sciences
University of Groningen
Nijenborgh 4, 9747 AG Groningen, The Netherlands}


\author{Toon Verstraelen}
\affiliation[TU Ghent]
{Center for Molecular Modeling
Ghent University
Tech Lane Ghent Science Park Campus A, 9052 Zwijnaarde, Belgium}
\email{toon.verstraelen@ugent.be}

\author{Jack D. Evans}
\affiliation[TU Dresden]
{Department of inorganic chemistry
Technische Universität Dresden
Bergstraße 66, 01062 Dresden, Germany}
\email{jack.evans@tu-dresden.de}


%%%%%%%%%%%%%%%%%%%%%%%%%%%%%%%%%%%%%%%%%%%%%%%%%%%%%%%%%%%%%%%%%%%%%
%% The document title should be given as usual. Some journals require
%% a running title from the author: this should be supplied as an
%% optional argument to \title.
%%%%%%%%%%%%%%%%%%%%%%%%%%%%%%%%%%%%%%%%%%%%%%%%%%%%%%%%%%%%%%%%%%%%%
\title[]
  {Mapping the complete thermodynamic landscape of gas adsorption for a responsive metal-organic framework}

%%%%%%%%%%%%%%%%%%%%%%%%%%%%%%%%%%%%%%%%%%%%%%%%%%%%%%%%%%%%%%%%%%%%%
%% Some journals require a list of abbreviations or keywords to be
%% supplied. These should be set up here, and will be printed after
%% the title and author information, if needed.
%%%%%%%%%%%%%%%%%%%%%%%%%%%%%%%%%%%%%%%%%%%%%%%%%%%%%%%%%%%%%%%%%%%%%
\abbreviations{IR,NMR,UV}
\keywords{American Chemical Society, \LaTeX}

%%%%%%%%%%%%%%%%%%%%%%%%%%%%%%%%%%%%%%%%%%%%%%%%%%%%%%%%%%%%%%%%%%%%%
%% The manuscript does not need to include \maketitle, which is
%% executed automatically.
%%%%%%%%%%%%%%%%%%%%%%%%%%%%%%%%%%%%%%%%%%%%%%%%%%%%%%%%%%%%%%%%%%%%%
\begin{document}

%%%%%%%%%%%%%%%%%%%%%%%%%%%%%%%%%%%%%%%%%%%%%%%%%%%%%%%%%%%%%%%%%%%%%
%% The "tocentry" environment can be used to create an entry for the
%% graphical table of contents. It is given here as some journals
%% require that it is printed as part of the abstract page. It will
%% be automatically moved as appropriate.
%%%%%%%%%%%%%%%%%%%%%%%%%%%%%%%%%%%%%%%%%%%%%%%%%%%%%%%%%%%%%%%%%%%%%
\begin{tocentry}


The surrounding frame is 9\,cm by 3.5\,cm, which is the maximum
permitted for  \emph{Journal of the American Chemical Society}
graphical table of content entries. The box will not resize if the
content is too big: instead it will overflow the edge of the box.


\end{tocentry}

%%%%%%%%%%%%%%%%%%%%%%%%%%%%%%%%%%%%%%%%%%%%%%%%%%%%%%%%%%%%%%%%%%%%%
%% The abstract environment will automatically gobble the contents
%% if an abstract is not used by the target journal.
%%%%%%%%%%%%%%%%%%%%%%%%%%%%%%%%%%%%%%%%%%%%%%%%%%%%%%%%%%%%%%%%%%%%%
\begin{abstract}
  To do.
\end{abstract}

%%%%%%%%%%%%%%%%%%%%%%%%%%%%%%%%%%%%%%%%%%%%%%%%%%%%%%%%%%%%%%%%%%%%%
%% Start the main part of the manuscript here.
%%%%%%%%%%%%%%%%%%%%%%%%%%%%%%%%%%%%%%%%%%%%%%%%%%%%%%%%%%%%%%%%%%%%%
\section{Introduction}

Advanced porous materials are poised for application.

It is often the flexibility of porous adsorbents that enables unparalleled performance in separation applications.

Responsive adsorbents, such as metal-organic frameworks, can significantly change pore structure and volume in response to stimuli.\cite{10.1021/acs.chemmater.5b00046}

One particularly challenging example is negative gas adsorption (NGA).

To completely describe the thermodynamics of responsive frameworks simulations of a fully flexible solid in presence of adsorbate are necessary.
Simulations, within this osmotic ensemble provide access to the grand canonical potential ($\Omega$).
This potential completely describes the thermodynamic features of the system with respect to the energies of the framework, the chemical potential of the adsorbed fluid, external pressure and temperature.
Atomistic simulations within the osmotic ensemble remain challenging because dynamics must be combined with particle insertion/deletion, which has only been applied to handful of systems.

Analysis of the grand canonical potential are often achieved using analytical descriptions of gas adsorption.
This has been particularly insightful the relative stability of two distinct and porous phases.
For example, the process of ``breathing'' can be directly seen for two different porous phases based on the difference in grand canonical potential, $\Omega$ (Figure~\ref{fgr:figure1}a,b).

In the present work we have employed state-of-the-art molecular simulations to produce a detailed picture of the entire landscape of grand canonical potential for changes in pressure, amount adsorbed and cell volume for DUT-49.
This complete thermodynamic analysis provides unprecedented insight into the phase stability under adsorption.
Our analysis recreates experimental observations, demonstrates the effects of temperature and outlines the kinetic and thermodynamics phases present during NGA.

% \begin{figure}[htb]
%   \includegraphics{../figures/figure1-01.png}
%     \caption{}
%     \label{fgr:figure1}
%   \end{figure}


Our methodology uses grand canonical Monte Carlo moves combined with molecular dynamics trajectories in the $(N,V,\sigma_{a}=0,T)$ ensemble, to effectively constraining the volume while allowing the cell shape to fluctuate. Subsequently, adsorption isotherms at each unit cell volume then are used to construct the grand canonical potential as Equation~{\ref{eqn:osmoticequation}}.

% \begin{equation}
%   \Omega\left(N_{\mathrm{host}}, p, \sigma_{a}=\mathbf{0}, T ; V\right)=F_{\mathrm{host}}\left(N_{\mathrm{host}}, T ; V\right)+p V-\int_{-\infty}^{\mu(p, T)} N_{\mathrm{guest}}\left(N_{\mathrm{host}}, \mu^{\prime}, \sigma_{a}=\mathbf{0}, T ; V\right) \mathrm{d} \mu^{\prime}
%   \label{eqn:osmoticequation}
% \end{equation}

The resulting potential at different methane gas pressures and cell volumes is displayed in Figure~\ref{fgr:figure1}c. To obtain accurate integrals of the number of adsorbed particles (third term in Equation~{\ref{eqn:osmoticequation}}), a large number of chemical potentials were considered.

This detailed potential landscape contains metastable and equilibrium states that change with  increasing gas pressure (Figure~\ref{fgr:figure2}). The system, at zero gas pressure, begins at equilibrium in the $op$ phase (large cell volume) where there is also a metastable $cp$ phase (small cell volume. The observed states correspond are in quantitative agreement with in situ studies.
Intermediate phases, between the $op$ and $cp$ phases, are observed by simulation and experiment further highlighting the accuracy of our proposed landscape.
  
  % \begin{figure}[htb]
  %   \includegraphics{../figures/figure2-01.png}
  %     \caption{}
  %     \label{fgr:figure2}
  %   \end{figure}

Simulations were completed for three temperatures (90$\,$K, 120$\,$ and 150$\,$) to provide insight to the thermodynamic trends responsible for the observations of different phases for DUT-49.
Particularly, temperature demonstrates a fascinating non-monotonic relationship with amount of gas released during the negative gas adsorption process.
The relative grand potential between the $op$ and $cp$ phases can be directly computed from these simulations, in addition to the energy barrier between these states (Figure~\ref{fgr:figure3}a). It is observed that the pressure window for stability of the $cp$ phase is correlated with temperature, due to the entropic effects of condensing gas to a smaller pore structure.
The maximum energy difference is also correlated with temperature.
Contrastingly, the minimum barrier height shows a weaker trend with temperature.
A temperature of 90 and 120$\,$K shows no energy barrier between the states.
We can alternatively consider these trends with respect to the difference in  adsorbed amount in the $op$ and $cp$ phases ((Figure~\ref{fgr:figure3}b), which represents the magnitude of negative gas adsorption. 
Interestingly, the $cp$ phase can become more favorable than the $op$ phase before the crossing point of the individual isotherms, represent by 0.
This is especially true for the.

%     \begin{figure}[htb]
%       \includegraphics{../figures/figure3-01.png}
%         \caption{}
%         \label{fgr:figure3}
%       \end{figure}

% The finally.

%       \begin{figure*}[htb]
%         \includegraphics{../figures/figure4-01.png}
%           \caption{}
%           \label{fgr:figure4}
%         \end{figure*}





%%%%%%%%%%%%%%%%%%%%%%%%%%%%%%%%%%%%%%%%%%%%%%%%%%%%%%%%%%%%%%%%%%%%%
%% The "Acknowledgement" section can be given in all manuscript
%% classes.  This should be given within the "acknowledgement"
%% environment, which will make the correct section or running title.
%%%%%%%%%%%%%%%%%%%%%%%%%%%%%%%%%%%%%%%%%%%%%%%%%%%%%%%%%%%%%%%%%%%%%
\begin{acknowledgement}
  J. D. E. acknowledges the support of the Alexander von Humboldt foundation and HPC platforms provided by a GENCI grant (A0070807069) and the Center for Information Services and High Performance Computing (ZIH) at TU Dresden.
\end{acknowledgement}

%%%%%%%%%%%%%%%%%%%%%%%%%%%%%%%%%%%%%%%%%%%%%%%%%%%%%%%%%%%%%%%%%%%%%
%% The same is true for Supporting Information, which should use the
%% suppinfo environment.
%%%%%%%%%%%%%%%%%%%%%%%%%%%%%%%%%%%%%%%%%%%%%%%%%%%%%%%%%%%%%%%%%%%%%
% \begin{suppinfo}

% This will usually read something like: ``Experimental procedures and
% characterization data for all new compounds. The class will
% automatically add a sentence pointing to the information on-line:

% \end{suppinfo}

%%%%%%%%%%%%%%%%%%%%%%%%%%%%%%%%%%%%%%%%%%%%%%%%%%%%%%%%%%%%%%%%%%%%%
%% The appropriate \bibliography command should be placed here.
%% Notice that the class file automatically sets \bibliographystyle
%% and also names the section correctly.
%%%%%%%%%%%%%%%%%%%%%%%%%%%%%%%%%%%%%%%%%%%%%%%%%%%%%%%%%%%%%%%%%%%%%
\bibliography{article}

\end{document}